\chapter{Commutative algebra}

\section{Rings}
\label{sec:rings}

\begin{prop}%
  \label{prop:preimage-of-prime-is-prime}
  The preimage of a prime ideal is also a prime ideal: given a ring map $\phi :
  A \to B$ and a prime ideal $I \subset B$, $\inv\phi(I) \subset A$ is also
  prime.
\end{prop}
\begin{proof}
  Let $J = \phi^{-1}(I)$, and consider $xy \in J$. Then $\phi(xy) =
  \phi(x)\phi(y) \in I$, so either $\phi(x) \in I$ or $\phi(y) \in I$, which in
  turn tells us that either $x \in J$ or $y \in J$.
\end{proof}

\begin{prop}%
  \label{prop:surjective-image-of-prime-is-prime}
  The surjective image of a prime ideal is also a prime ideal: given a
  surjective ring morphism $\sigma : A \twoheadrightarrow B$ and a prime ideal
  $I \subset A$, $\sigma(I) \subset B$ is also prime.
\end{prop}
\begin{proof}
  This is a simple variant of the argument for
  \Cref{prop:preimage-of-prime-is-prime}. Let $J = \sigma(I)$, and consider $xy
  \in J$. By surjectivity, there exist $a$ and $b$ such that
  \begin{align*}
    \sigma(a) &= x\\
    \sigma(b) &= y
  \end{align*}
  yielding $\sigma(ab) = xy$. Then $ab \in I$, which implies that (wlog) $a \in
  I$ since $I$ is prime. Then $\sigma(a) = x \in J$, so that $J$ is also prime.
\end{proof}

\begin{prop}
  \label{prop:inclusion-preserving-bijection-quotient}
  Let $I$ be an ideal of $A$. Consider the natural map
  \[\pi : A \to A/I.\]
  Then $\inv\pi$ gives an inclusion-preserving bijection between prime ideals of
  $A/I$ and prime ideals of $A$ containing $I$.
\end{prop}
\begin{proof}
  This is one form of the third isomorphism theorem (TODO REF).

  \medskip\noindent
  For one inclusion, we apply \Cref{prop:preimage-of-prime-is-prime}, noting
  that (after convincing oneself that the map preserves inclusions) the preimage
  of a prime ideal $P \subset A/I$ must contain $I$ since $(0) \subset P$ and
  $\inv\pi((0)) = I$. TODO
\end{proof}

\begin{prop}
  The map $\phi: k[x] \to k[x,\epsilon]/(\epsilon^2)$ sending $ x \mapsto
  x+\epsilon $ maps \[ f(x) \mapsto f(x) + \epsilon f'(x). \]
\end{prop}

\section{Tensor products of modules}
\label{sec:tensor-products-of-modules}

\section{Operations on modules}
\label{sec:operations-on-modules}

\section{Localization}
\label{sec:localization}