\chapter{Fourier analysis}

\section{Fourier expansions}
\label{sec:fourier-expansions}

Let $g : \bc \to \hat\bc$ be a continuous function with period $1$.

The $n$th \emph{Fourier coefficient} $a_n(y)$ is
\[ a_n(y) = \hat g(n) = \int_0^1 g(z) exp(-2\pi i n z) dx \]

Then we have the \emph{Fourier expansion}
\[ g(z) = \sum_{n=-\infty}^\infty a_n(y) exp(2\pi inz) \]

\section{Meromorphicity}
\label{sec:meromorphicity}

The \emph{nome} is a common building block for interesting functions.

\[ q = q(z) := exp(2 \pi i z) \]

Let $g$ be meromorphic in the notation of the previous section. Then there
exists a unique meromorphic $G : \bc^\times \to \hat\bc$ such that $g(z) =
G(q)$ (TODO why?): in other words, a period-$1$ meromorphic function of $z$ is in fact a
function of $q(z)$.

Note that $G$ has a removable singularity at $0$, so, by Theorem ???, $G$
extends to a meromorphic function on $\bc$ iff
\[ \lim_{q\to0} G(q) |q|^m = 0\] for some $m$. What does it mean for $q$ to go
to $0$?

\begin{align*}
  q \to 0 &\implies exp(2\pi i(x+iy)) \to 0\\
          &\implies exp(2\pi i x) e^{-2 \pi y} \to 0\\
          &\implies y \to \infty
\end{align*}

so we have \( g(z)|q|^m \to 0\) as \( g(z) exp(-2\pi m y) \to 0 \), so we need

\[ \text{as } \Im(z) \to \infty, \exists m\;\; |g(z)| < exp(2\pi my) \]

The meromorphy of $G(q)$ at $0$ thus requires $\Im(z)\to\infty$, in which case
we say $g$ is \emph{meromorphic at $i\infty$}. Then $G$, being meromorphic at
$0$, has a Laurent series expansion

\[ g(z) = G(q) = \sum_{n=-m}^\infty c_nq^n = \sum_{n=-m}^\infty c_n e^{2\pi inz} \]

Here $m$ is the order of the pole of $G$ at $0$. However, we also have a Fourier
expansion

\[ g(z) = \sum_{n=-\infty}^\infty a_n(y) e^{2\pi inz} \]

and, equating coefficients,

\begin{align*}
  a_n(y) &= c_n &\text{ for } n\ge -m\\
  a_n(y) &= 0   &\text{ for } n < -m
\end{align*}
