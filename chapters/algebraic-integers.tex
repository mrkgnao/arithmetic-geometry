\chapter{Algebraic integers}

\epigraph{Pellentesque condimentum, magna ut suscipit hendrerit, ipsum augue
  ornare nulla, non luctus diam neque sit amet urna.}{Someone}

\section{Appetizer: Fermat's theorem on sums of two squares}%
\label{sec:appetizer-fermats-theorem-on-sums-of-two-squares}

\begin{question}
  Does the equation
  \begin{align}p = a ^ 2 + b ^ 2 &&(p \text{ prime})\end{align} have nontrivial
  solutions with integer \(a\) and \(b\)?
\end{question}

\begin{paragraph}{A bit of history}
  This section is salvaged from a stray {\LaTeX} file found in
  an old \texttt{/home} folder. I wrote this for a friend at Canada/USA
  Mathcamp, as part of my usual Number Theory Indoctrination™ Service.
\end{paragraph}

\medskip\noindent
One start is to look at the equation mod \(4\). Since squares
are always either \(0\) or \(1\) modulo \(4\) (work this out for yourself if
it's not familiar) \(p\) can only be \(1\) mod \(4\). \footnote{\(0\) and \(2\)
  are, uh, not really possibilities, except in the case \(2 = 1^2 + 1^2\).}
Keeping this in mind, we restrict our attention to primes \(\equiv 1\Mod 4\).

We notice that a nontrivial "factorization" of \(p\) of the form

\[p=(a+ib)(a-ib) a, b\in{\mathbf Q}\]

gives us an expression of the form we want, if we are willing to expand our
notion of what factorization means beyond its usual meaning in \({\mathbf Z}\).
In fact, the approach that we follow is to look at how different primes factor
or "split" in the ring \({\mathbf Z}[i]\), which is the ring of numbers of the
form

\[ {\mathbf Z}[i] = \{a+ib : a, b\in{\mathbf Z}\} \]

We will obtain a complete description of how integer primes split in this ring,
the \newTerm{ring of Gaussian integers}, and thus prove the following:

\begin{theorem*}[Fermat]
  A prime number can be expressed as a sum of two integer squares iff it is
  congruent to \(1\) mod \(4\).
\end{theorem*}

\subsection{\({\mathbf Z}[i]\) as a lattice}

Notice that it can often be worthwhile to think of \({\mathbf Z}[i]\)
geometrically, as a subset of the complex plane corresponding to all the complex
numbers with integer real and complex parts. This gives \({\mathbf Z}[i]\) the
structure of a (square) \newTerm{lattice}, which has an obvious meaning that we
will not work hard to rigorize for now: think of the points in the plane with
integer coordinates, where \( (x,y) \) corresponds to \( x+iy \).

\begin{figure}
  \centering
  \begin{tikzpicture}   
    \begin{scope}
      \clip (-5,-5) rectangle (5cm,5cm); % Clips the picture... 
      \draw[style=help lines,dashed] (-14,-14) grid[step=2cm] (14,14); 
      \foreach \x in {-7,-6,...,7}{ % Two indices running over each 
        \foreach \y in {-7,-6,...,7}{ % node on the grid we have drawn 
          \node[draw,circle,inner sep=2pt,fill] at (2*\x,2*\y) {};
        }
      }   
    \end{scope}   
  \end{tikzpicture} 
\end{figure}

\begin{aside}
  This has very deep applications once we start considering more interesting
  number fields: here we are looking at \({\mathbf Q}(i)\) and its \newTerm{ring
    of integers} \({\mathbf Z}[i]\), and \({\mathbf Q}(i)\) is basically the
  nicest number field in existence. We will see that arguments using the
  "geometry of numbers", also called \newTerm{Minkowski theory}, are very powerful, and
  will (for instance) enable us to prove the Dirichlet unit theorem.
\end{aside}
\subsection{Preliminaries}

First, we review a few definitions from ring theory.

% <!-- \begin{defn} --> An element \(r\) of a ring \(R\) is \newTerm{prime} if,
%   whenever it divides a product \(ab\), it must divide either \(a\) or \(b\).
%   <!-- \end{defn} -->

% <!-- \begin{defn} --> An element \(r\in R\) is \newTerm{irreducible}<label
%   for="sn-demo" class="margin-toggle sidenote-number"></label><input
%   type="checkbox" id="sn-demo" class="margin-toggle"/><span
%   class=sidenote>\newTerm{irred} if I'm lazy</span> if the only way to write
%   it as a product of elements is
%   \[r=u\cdot r_1\] where \(u\) is a unit, and \(r_1\) is \(u_1r\) for some
%   unit \(u_1\). <!-- \end{defn} -->

For example, \(2\) is not irred in \({\mathbf Z}[i]\), since \(2={(1+i)}^2\),
but it is an irred in \({\mathbf Z}\), since the only ways to write it as a
product are silly things like \((-1)\cdot (-2)\) and \(-2\) is manifestly a unit
times \(2\).

\begin{definition}
  A ring \(R\) is \newTerm{Euclidean} if there exists a function \[\nu :
    R\setminus\{0\}\to{\mathbf Z} ^ {\geq 0},\] called the \newTerm{Euclidean
    valuation}, such that we can perform a "division algorithm" in the ring
  using \(\nu\).
\end{definition}
That is, for any \(a, b\in R\), there exist \(q, r\in R\) such
that\footnote{Note that the valuation of \(0\) is not defined, at least per this
  definition. However, setting \(\nu(0)=0\) seems to work just fine.}


\[a=bq+r, \nu(r) < \nu(b).\]

For instance, \({\mathbf Z}\) is an Euclidean domain with valuation \(\nu(r) =
|r|\).

\begin{definition}
  A \newTerm{principal ideal domain}, or \newTerm{PID} for short, is a ring
  where all ideals are generated by a single element (all ideals are
  \newTerm{principal}).
\end{definition}

\begin{definition}
  A \newTerm{unique factorization domain} (often \newTerm{UFD} for short, or
  sometimes \newTerm{factorial} ring or domain) is a ring where elements can be
  uniquely factored into irreducible elements. That is, for all \(r\in R\),
  there exists a factorization

  \[ r=u\cdot p_1 ^ {a_1}p_2 ^ {a_2}\cdots p_k ^ {a_k} \]

  where \(u\) is a unit, and the \(p_i\) and \(a_i\) are unique, up to
  reordering of the \(p_i\).
\end{definition}

Now, we have:

\begin{theorem}
  Euclidean domain \(\implies\) PID.
\end{theorem}
\begin{proof}
  Consider an arbitrary ideal \(I\) of our ring \(R\). Choose a smallest nonzero
  element \(w\) from \(I\) (where "smallest" refers to the valuation being the
  least).

  Any other element of \(I\) can be written as

  \[a = wq + r\]

  where \(\nu(r) < \nu(w)\). But there is no such nonzero element, by the
  assumption that \(w\) has the smallest valuation of any element of \(I\). We
  conclude that \(r=0\). Hence \(w\) divides every other element of \(I\), so
  \(I=(w)\).
\end{proof}

\begin{atheorem}[Very Useful]
  PID \(\implies\) UFD.
\end{atheorem}

\begin{proof}
  The proof is too messy to include here. Some references are provided in the
  margin. A quick google turned up this pdf. ProofWiki also has an incomplete
  proof.
\end{proof}


% http://www.maths.nuigalway.ie/MA416/section4-2.pdf.
% https://proofwiki.org/wiki/Principal_Ideal_Domain_is_Unique_Factorization_Domain

It's worth taking a moment to do the following (easy) exercise:

\begin{exercise}
  In any integral domain, primes are irreducible.
\end{exercise}

There is, in fact, a very important partial converse:

\begin{exercise}[and Fact]
  In a UFD, irreducibles are prime.
\end{exercise}

This will be helpful shortly. In fact, the only reason why we went to all the
trouble with PIDs and so on is so that we could state this fact!

\subsection{\({\mathbf Z}[i]\) is a UFD}

\begin{prop}
  The function \(\nu: {\mathbf Z}[i]\to{\mathbf Z}^{\geq 0}\) defined by

  \[ \nu(a + ib) = a^2 + b^2 = | a + ib |^2 \]

  is an Euclidean valuation on the ring \({\mathbf Z}[i]\).
\end{prop}

\begin{proof}
  Given \(a,b\in{\mathbf Z}[i]\), we need to show that there exist
  \(q,r\in{\mathbf Z}[i]\) such that

  \begin{align}a=bq+r &&|r|^2 < |b|^2\end{align}

  This is equivalent to \(\frac ab - q = \frac r b.\)

  We need to find a \(b\) such that
  \[\left|\frac ab - q\right| = \left| \frac rb\right|< 1.\]

  Why is this possible? Notice that the number \(\frac ab\) lies in some square
  of the lattice formed by the Gaussian integers in the complex plane. So there
  will always be some lattice point within (at most) half the length of a
  diagonal of a lattice square from \(\frac ab\). This suffices, since that
  length is \(\frac{\sqrt 2}2 < 1\).
\end{proof}
If \(p=4n+1\) is a prime, then the congruence \[-1\equiv x^2\Mod p\] has a
solution: indeed, by Wilson's theorem,

\begin{align}
  -1\equiv (p-1)! &= (1\cdot2\cdots{(2n)})\cdot((2n+1)\cdot(2n+2)\cdots{(4n)}) \\
                  &\equiv (1\cdot2\cdots{(2n)})\cdot((-2n)\cdot(-2n+1)\cdots{(-1)}) \\
                  &= (2n)!\cdot{(-1)^{2n}(2n)!} \\
                  &= {[(2n)!]}^2 \Mod p
\end{align}

so taking \(x=(2n)!\) works.

Now we are ready to prove the main theorem. We restate it here:

\begin{theorem}[Fermat]
  A prime number can be expressed as a sum of two integer squares iff it is
  congruent to \(1\) mod \(4\).
\end{theorem}

\begin{proof}
  It now suffices to show that a prime \(p\in{\mathbf Z}\) does not remain prime
  in \({\mathbf Z}[i]\) if \(p\equiv 1\Mod 4\). When that is done, we have a
  nontrivial factorization

  \[ p=\alpha\cdot\beta \]

  and taking norms (i.e. valuations) of both sides gives

  \[ p^2 = (a^2 + b^2) \cdot\nu(\beta) \]

  where we write \(\alpha = a+ib\). Now, since the factorization is nontrivial,
  neither \(\alpha\) nor \(\beta\) are units, and hence we have \( p = a^2 + b^2
  \), and we're done.

  But for primes congruent to \(1\) modulo \(4\), we have our lemma which states
  that there exists \(x\) such that \(x^2+1\equiv 0\Mod 4\). So \(\divides p {x^2+1 =
  (x+i)(x-i)}\), but it does not divide either of the factors on the right (it
  doesn't divide either of their imaginary parts). So \(p\) is not prime in
  \({\mathbf Z}[i]\).

  Now we use all the algebra we did: since \({\mathbf Z}[i]\) is a UFD, our Fact
  from before tells us that \(p\) not prime implies that \(p\) is not
  irreducible. So \(p\) has some factorization

  \[p=\alpha\cdot\beta\]

  in \({\mathbf Z}[i]\), and we're done.
\end{proof}

\section{Integrality}

Let $\ringExtension A B$ be an extension of rings.

\begin{definition}
  An element $b \in B$ is called integral over $A$ if it satisfies a monic equation
  \[ x^n + a_{n-1}x^{n-1} + \cdots + a_0 = 0 \]
  with $a_i \in A$.
\end{definition}

\begin{definition}
  $B$ is integral over $A$ if all $b\in B$ are integral over $A$.
\end{definition}

\begin{definition}
  The characteristic polynomial of an element $\alpha$ will be denoted $\chi_\alpha$.
\end{definition}

\begin{definition}
  The minimal polynomial of an element $\alpha$ will be denoted $\mu_\alpha$.
\end{definition}

\subsection{Basic properties}

Integrality is stable under the ring operations: one would like the integrality
of $a$ and $b$ to imply the integrality of $a + b$ and $ab$. This does hold,
and we will be able to see it once we recast the notion of integrality in terms
of commutative algebra.

\begin{lemma}
  Let $A = (a_{ij})$ be an $r\times r$ matrix over a ring $R$, and let \[A^\ast
  = ({a_{ij}}^\ast) = ((-1)^{i+j} \det A_{ij})\] be the cofactor matrix of $A$.
  Writing $\Delta$ for $\det A$, \[A{A^\ast} = {A^\ast}A = \Delta I_r \] which
  implies that, given $x = (x_1,\ldots,x_r)$, \[Ax = 0 \implies \Delta x = 0.\]
\end{lemma}

\begin{theorem}[Integrality and finiteness]{\label{module-integrality}}
  A finite number of $b_i$ are integral over $A$ $\iff$ the ring
  $A[b_1,\ldots,b_n]$ is finitely generated as an $A$-module.
\end{theorem}
\begin{proof}
  Let $b\in B$ be integral over $A$, with $\beta(x) \in A[x]$ a monic
  polynomial of degree $n$ satisfying $\beta(b) = 0$. For arbitrary $f\in
  A[x]$, we can use the division algorithm in $A[x]$ to get \[f = g\beta + r,\]
  where $\rho := \deg r < n$.  Then \[f(b) = g(b)\beta(b) + r(b) = r(b) =
  a_\rho b^\rho + \cdots + a_0\] so $A[b]$ is generated as an $A$-module by $1,
  b, \ldots, b^\rho$.

  \medskip\noindent For the converse, let $B$ be a finitely generated
  $A$-module with generators $t_1,\ldots,t_n$. Any $\lambda \in B$ satisfies \[
  \lambda t_i = \sum_j a_{ij} t_j \] for some coefficients $a_{ij} \in A$.
  Writing $A = (a_{ij})$ and $t = (t_1,\ldots,t_r)$, $(\lambda
  t_1,\ldots,\lambda t_r) = \lambda t$ and \[ \left( \sum_j a_{1j} t_j, \cdots,
  \sum_j a_{rj} t_j \right) = At \]
  which gives
  \[ (\lambda I_r - A)t = 0 \]
  implying, by our lemma, that
  \begin{equation}%
    \label{integrality-finiteness-penult-eqn}
    \det(\lambda I_r - A) \cdot t_i = 0 
  \end{equation}
  for all $i$. Now, we know that the $t_i$ form a generating set, so, in
  particular, $1$ can be written as a linear combination of the $t_i$. This
  allows us to combine the system of equations
  \ref{integrality-finiteness-penult-eqn} into \[\det(\lambda I_r - A) = 0 \]
  which is a monic equation for $\lambda$ over $A$. (Isn't this $\chi_A(\lambda)$?)
\end{proof}

\begin{corollary}{\label{integrality-of-sums-products}}
  If $a$ and $b$ are integral over $A$, so are $a + b$ and $ab$.
\end{corollary}

\begin{theorem}[Integrality is transitive]{\label{integrality-trans}}
  Consider ring extensions $A \subseteq B \subseteq C$. $A \subseteq B$ integral
  and $B \subseteq C$ integral $\iff$ $A \subseteq C$ integral.
\end{theorem}
\begin{proof}
  $A \subseteq C$ integral implies $A \subseteq B$ integral, since every
  element of $B$ is an element of $C$. Similarly, since every element $c\in C$
  satisfies a monic polynomial in $A[x]$ and $A[x] \subseteq B[x]$, it also
  implies $B \subseteq C$ integral. Hence the integrality of the composite
  extension $A\subseteq C$ implies that the subextensions are also integral.

  \medskip\noindent Conversely, let $c\in C$ satisfy \[ c^n + b_{n-1}c^{n-1} +
  \cdots + b_0 = 0 \] where $b_i\in B$. Let $R = A[b_1,\cdots,b_n]$. Then
  $R \subseteq {R[c]}$ is finite since $c$ is integral over $A$, and
  $A \subseteq R$ is finite since $A \subseteq B$ is. Hence the
  composite extension $A \subseteq R[c]$ is finitely generated as an
  $A$-module and is thus integral.
\end{proof}

\subsection{Integral closures}

\begin{definition}
  Given a ring extension $A \subseteq B$, the \newTerm{integral closure} $\bar A$
  is the ring consisting of elements $b\in B$ integral over $A$.
\end{definition}

\begin{definition}
  The integral closure of an integral domain $A$ in its field of fractions
  $K(A)$ is called the \newTerm{normalization} of $A$.
\end{definition}

\begin{definition}
  If $\bar A = A$ in the extension $A \subseteq B$, $A$ is said to be
  \newTerm{integrally closed} in B.
\end{definition}

When the data of $B$ is omitted, as in ``$A$ is integrally closed'', one takes $B = K(A)$.

\begin{prop}
  UFDs are integrally closed.
\end{prop}
\begin{proof}
  Let $p/q$ be a reduced fraction in $K(A)$ integral over $A$:
  \[ (p/q)^n + a_{n-1}(p/q)^{n-1} + \cdots + a_0 = 0 \]
  Clearing denominators yields
  \[ p^n     + a_{n-1} q p^{n-1}  + \cdots + q^n a_0 = 0 \]
  which gives $\divides q p$. This implies, by the hypothesis that the fraction was in lowest terms,
  that $q = 1$. So $p/q = p$ is an element of $A$.
\end{proof}

\medskip\noindent
Fix a domain $A$ integrally closed in $K := K(A)$. Let $\ringExtension L K$ be a
finite extension, and $B$ the integral closure of $A$ in $L$. This is the
\newTerm{$AKLB$ diagram}:

\[
  \begin{tikzcd}
    K \arrow[r, hook] & L \\
    A \arrow[u] 
      \arrow[r]       & B \arrow[u]
  \end{tikzcd}
\]

Many of our results will seek to describe how certain properties behave under
the maps in the diagram.

\begin{remark}
  Notice that $K(B) = L$. Indeed, $B \subset L$ so $K(B) \subset L$, and the
  result above shows that $L \subset K(B)$ (set-theoretically, $L \subset B
  \times A \subset B \times B$).
\end{remark}

\begin{theorem}
  Any element $l \in L$ can be expressed as $b/a$ for some $b \in B$ and $a \in A$.
\end{theorem}

\begin{proof}
  Consider an element $l \in L$. The minimal polynomial $\mu_l$ of $l$ over $K$
  gives rise to a polynomial over $A$
  \[ a_nl^n + a_{n-1}l^{n-1} + \cdots + a_0 = 0 \] by clearing denominators. Now
  observe that $\ell := a_nl$ is integral over $A$: multiplying by $a_n^{n-1}$
  gives an equation of the form
  \[ \ell^n + a_{n-1}'\ell^{n-1} + \cdots + a_0' = 0. \] This shows that taking
  $b/a$ = $\ell/a_n$ works.
\end{proof}

\begin{theorem}
  $l \in L$ is integral over $A$ iff its minimal polynomial $\mu_l$ over $K$ has
  coefficients in $A$.
\end{theorem} 

\begin{proof}
  If $\mu := \mu_l(x)$ has coefficients in $A[x]$ then the integrality of $l$
  over $A$ follows from the definition (recall that minimal polynomials are
  monic). 
  
  \medskip\noindent Consider now the case of an integral element $l$ with
  minimal polynomial $\mu \in K[x]$. From integrality over $A$ we know that $l$
  is a root of some $g(x) \in A[x]$. Then $\divides \mu g$ in $K[x]$, so all
  zeros of $\mu$ are zeros of $g$ and hence integral over $A$.  \npar By
  Viet\`a, the coefficients $a_i$ are given by elementary symmetric polynomials
  in the roots and are hence, by \Cref{integrality-of-sums-products}, integral
  over $A$ themselves. Since the $a_i$ are elements of $K$, their integrality
  implies that they are actually elements of $A$.
\end{proof}

\begin{remark}
  It might bear explaining why the last step of the proof makes sense: why does
  the integrality of the $a_i$ over $A$ mean that they are actually elements of
  $A$? Every $A$-fraction $p/q$ satisfies a linear polynomial $qx-p = 0$, whose
  monicity gives $q = 1$, implying $p/q = p \in A$. 
\end{remark}

\section{The trace and the norm}

Given $x \in L$, multiplication by $x$ determines an endomorphism
\[ T_x : \alpha \mapsto x\alpha \] of the $K$-vector space $L$. We define the
trace and norm maps
\begin{align*}
  \Tr z &= \tr T_z\\
  \Nm z &= \det T_z
\end{align*}

Let $n = \fieldExtensionDegree L K$. The characteristic polynomial
\begin{align*}
  \chi_z(t) &= \det (tI - T_z)\\
            &= t^n - a_1t^{n-1} + \cdots + (-1)^n a_n \in K[t]
\end{align*}

contains coefficients $a_1 = \Tr z$ and $a_n = \Nm z$.

\begin{remark}
  If this isn't immediately clear, think Viet\`a. (This will be one of the
  recurring themes throughout this chapter.)
\end{remark}

Since $T_{x+y} = T_x + T_y$ and $T_{xy}$ = $T_x \triangleleft T_y$, the trace and norm are
in fact homomorphisms

\begin{alignat*}{2}
  \Trm &: L      &&\to K \\
  \Nmm &: L^\ast &&\to K^\ast
\end{alignat*}

\section{Galois-theoretic interpretations}

Fix an algebraic closure $\bar{K} = K^{alg}$ of $K$.

\begin{prop}
  If $\ringExtension L K$ is separable, letting $\sigma : L \to \bar{K}$ vary
  over the $K$-embeddings of $L$ into $\bar{K}$, we have
  \begin{enumerate}
  \item \label{chi-z-product-exp} $\chi_z(t) = \prod_\sigma (t - \sigma z)$
  \item $\Tr z = \sum_\sigma \sigma z$
  \item $\Nm z = \prod_\sigma \sigma z$
  \end{enumerate}
\end{prop}

\begin{proof}
  Let $d=[L:K(z)]$. The characteristic polynomial is a power \[\chi_z = \mu_z^d
  \] of the minimal polynomial \[ \mu_z(t) = t^m + c_1 t^{m-1} + \cdots + c_m
  \] of $z$. To prove this, we consider the basis $x^0, x^1, \ldots, x^{m-1}$ of
  $\ringExtension {K(z)} K$. If $\alpha_1, \ldots, \alpha_d$ is a basis for
  $\ringExtension L {K(z)}$, then the $dm$ elements 

  \[
    \begin{array}{cccc}
      \alpha_1 &\alpha_1 x &\ldots &\alpha_1 {x^{m-1}} \\
      \alpha_2 &\alpha_2 x &\ldots &\alpha_2 {x^{m-1}} \\
      \vdots   &\vdots     &\ddots &\vdots \\
      \alpha_d &\alpha_d x &\ldots &\alpha_d {x^{m-1}} 
    \end{array}
  \]

  form a basis $\hb$ for the extension $\ringExtension L K$. The action of $T_x$ on each row
  is 

  \begin{alignat*}{2}
    &\alpha_i         &&\mapsto \alpha_i x \\
    &\alpha_i x       &&\mapsto \alpha_i x^2 \\
    &{}               &&\cdots\\
    &\alpha_i x^{m-2} &&\mapsto \alpha_i x^{m-1} \\
    &\alpha_i x^{m-2} &&\mapsto \alpha_i x^{m-1} \\
    &\alpha_i x^{m-1} &&\mapsto \alpha_i x^m \\
    &{}               &&= -\alpha_i (c_1 x^{m-1} + \cdots + c_m) \\
    &{}               &&= \sum_{j=1}^{m-1} (-c_j) (\alpha_i x^{m-j})
  \end{alignat*}

  Now we can write the matrix of $T_x$ with respect to $\hb$. It will only have
  blocks along the diagonal: each will correspond to an $\alpha_i$, and they
  will all be identical and equal to

  \[ M_z =
  \begin{bmatrix}
    0      & 1        & 0        & \cdots & 0 \\
    0      & 0        & 1        & \cdots & 0 \\
    \vdots & \vdots   & \vdots   & \ddots & \vdots \\
    0      & 0        & 0        & \cdots & 1 \\
    -c_m   & -c_{m-1} & -c_{m-2} & \cdots & -c_1
  \end{bmatrix}
  \]

  The characteristic polynomial of $M_z$ is just $\mu_z(t)$, and there are $d$ such blocks, so that 
  \[ \chi_z(t) = \det(tI - T_z) = (\det (tI - M_z))^d = \mu_z(t)^d. \]

  We need a short Galois-theoretic lemma here.
  
  \begin{lemma}
  The set $\hom_K(L,\bar K)$ is partitioned by the equivalence relation \[
  \sigma \sim \tau \iff \sigma z = \tau z \] into $m$ equivalence classes of
  $d$ elements each.
  \end{lemma}

  \begin{proof}
  There are $n$ $K$-embeddings of a degree-$n$ separable extension
  $\ringExtension L K$ into $\bar K$, by the primitive element theorem.
  Applying this in turn to the extensions $\ringExtension L K$ and
  $\ringExtension L {K(z)}$, we find that there are
  
  \begin{itemize} 
    \item $md$ embeddings $L \hookrightarrow \bar K$ that fix $K$ 
    \item $d$ embeddings $L \hookrightarrow \bar K$ that fix $K(z)$;
      equivalently, these are $K$-embeddings that also fix the element $z\in L$ 
  \end{itemize}

  Each of the $md$ $K$-embeddings $\sigma \in \hom_K(L,\bar K)$ is a
  member of an equivalence class corresponding to a choice $z_i = \sigma z$ of
  a conjugate of $z$ to generate $K(z)$ with (there are $m$ of those). For
  every such conjugate, the equivalence class comprises $d$ $K(z_i)$-embeddings
  of $L$ into $\bar K$.

  Note that the second point uses the fact that the algebraic closure
  ${K(z)}^{alg} = K^{alg} = \bar K$.
  \end{proof}

  Now we are free to choose a system of representatives $\sigma_1,\ldots,\sigma_m$ and write
  \[ \mu_z(t) = \prod_{i=1}^m (t - \sigma_i z) \]

  and, further,

  \begin{align*}
    \chi_z(t) = (\mu_z(t))^d &= \prod_{i=1}^m (t-\sigma_i z)^d \\
                             &= \prod_{i=1}^m \prod_{\sigma\sim\sigma_i} (t-\sigma_i) \\
                             &= \prod_{\sigma} (t-\sigma z)
  \end{align*}
  
  This proves \Cref{chi-z-product-exp}, which easily implies the others by a
  straightforward application of Viet\`a's formulas.
\end{proof}

\begin{theorem}{\label{mul-trace-norm}}
  For a tower of finite extensions $K \subseteq L \subseteq M$, we have

  \begin{alignat*}{2}
    \Trmg L K &\triangleleft \Trmg M L &&= \Trmg M K\\
    \Nmmg L K &\triangleleft \Nmmg M L &&= \Nmmg M K
  \end{alignat*}
\end{theorem}

\begin{proof}
  First, we assume that $\ringExtension M K$ is separable. Let $m =
  \fieldExtensionDegree L K$. The set $\hom_K(M,\bar K)$ is partitioned by the
  relation \[ \sigma \sim \tau \iff \restr{\sigma}{L} = \restr{\tau}{L} \] into
  $m$ equivalence classes (of $[M : L]$ elements each).

  \medskip\noindent
  Given representatives $\iseqi \sigma m$, the $K$-embeddings
  of $L$ into the algebraic closure $\bar K$ are \[\kembk L = \left\{
  \restr{\sigma_1}{L}, \ldots, \restr{\sigma_m}{L} \right\},\] and we can then
  compute
  \begin{align*}
    \Trmg M K x 
          &= \sum_{\sigma \in \kembk{M}} \sigma x \\ 
          &= \sum_{\left[\sigma_i\right] \in \kembk{L}} 
             \left(\sum_{\sigma \sim \sigma_i} \sigma x\right) \\
          &= \sum_{i = 1}^m \sum_{\sigma \sim \sigma_i} \sigma x \\
          &= \sum_{i = 1}^m \Trmg {\sigma_i M}{\sigma_i L} {\sigma_i x} \\
          &= \sum_{i = 1}^m \sigma_i \left(\Trmg M L x\right) \\
          &= \Trmg L K \left(\Trmg M L x\right)
   \end{align*}
   which is the desired expression for the trace map in $\ringExtension M K$.
\end{proof}

\section{Integral bases}

\begin{definition}
  The \newTerm{discriminant} of a basis $\alpha_i$ of a separable extension
  $\ringExtension L K$ is defined by
  \[ d(\alpha_1,\ldots,\alpha_n) = \det((\sigma_i\alpha_j))^2 \] where the
  $\sigma_i$ are the $K$-embeddings $L \hookrightarrow {\bar{K}}$.
\end{definition}

From the Galois-theoretic expression for the trace, we find \[ \Tr
  \alpha_i\alpha_j = \sum_\sigma \sigma\left(\alpha_i\alpha_j\right) = \sum_k
  (\sigma_k \alpha_i) (\sigma_k \alpha_j), \] we can write the matrix $\Tr
  \alpha_i\alpha_j$ as a product \[ \left(\Tr \alpha_i\alpha_j \right) =
  \left(\sigma_k\alpha_i\right)^t \left(\sigma_k \alpha_j\right) \]

This gives us an alternative expression for the discriminant.

\begin{prop}
  For a separable extension $\ringExtension L K$ with basis $\iseqi\alpha n$,
  \[ d(\iseqi \alpha n) = \det \left(\Tr \alpha_i\alpha_j \right). \]
\end{prop}

\begin{prop}
  If $\ringExtension L K$ has a basis of the form
  $\theta^0,\theta^1,\ldots,\theta^{n-1}$, one has 
  
  \[ d(\theta^0,\theta^1,\ldots,\theta^{n-1}) = \prod_{i<j} (\theta_i -
  \theta_j)^2 \] 

  \medskip\noindent
  where $\theta_i = \sigma_i \theta$, with $\sigma_i$ ranging over $\kembk L$.
\end{prop}

\begin{proof}
  \begin{align*}
    \begin{vmatrix}
      1      & \theta_1 & \theta_1^2 & \cdots & \theta_1^{n-1}\\
      1      & \theta_2 & \theta_2^2 & \cdots & \theta_2^{n-1}\\
      \vdots & \vdots   & \vdots     & \ddots & \vdots\\
      1      & \theta_n & \theta_n^2 & \cdots & \theta_n^{n-1}
    \end{vmatrix}
    &=
    \begin{vmatrix}
      1      & 0                   & 0          & \cdots & 0 \\
      1      & \theta_2 - \theta_1 & \theta_2(\theta_2 - \theta_1) 
             & \cdots & \theta_2^{n-2}(\theta_2 - \theta_1)\\
      \vdots & \vdots   & \vdots     & \ddots & \vdots\\
      1      & \theta_n - \theta_1 & \theta_n(\theta_n - \theta_1) 
             & \cdots & \theta_n^{n-2}(\theta_n - \theta_1)\\
    \end{vmatrix}\\
    &= 
    \begin{vmatrix}
      \theta_2 - \theta_1 & \theta_2(\theta_2 - \theta_1) &
      \cdots & \theta_2^{n-2}(\theta_2 - \theta_1)\\
      \vdots   & \vdots     & \ddots & \vdots\\
      \theta_n - \theta_1 & \theta_n(\theta_n - \theta_1) &
      \cdots & \theta_n^{n-2}(\theta_n - \theta_1)\\
    \end{vmatrix}\\
    &=
    \begin{vmatrix}
      1      & \theta_2 & \theta_2^2 & \cdots & \theta_2^{n-2}\\
      1      & \theta_3 & \theta_3^2 & \cdots & \theta_3^{n-2}\\
      \vdots & \vdots   & \vdots     & \ddots & \vdots\\
      1      & \theta_n & \theta_n^2 & \cdots & \theta_n^{n-2}
    \end{vmatrix} \cdot \prod_i (\theta_i - \theta_1)
  \end{align*}
  We recursively expand the new determinant to arrive at \[ \prod_j \prod_{i>j}
  (\theta_i - \theta_j) = \prod_{1\le i<j\le n} (\theta_j - \theta_i)\]
  which the reader may recognise as the Vandermonde determinant.
\end{proof}

\begin{prop}%
  \label{trace-form-bilinear}

  For $\ringExtension L K$ a separable extension with basis $\alpha_i$, the
  function \[ \left\langle x,y\right\rangle = \Tr{xy}\] yields a nondegenerate
  $K$-bilinear form on $L$.
\end{prop}

\begin{corollary} For $\ringExtension L K$ and $\alpha_i$ as above,
  \[d(\alpha_1,\ldots,\alpha_n) \neq 0.\]
\end{corollary}

\begin{proof}
  The form has matrix \[ M = \Tr {(\alpha_i\alpha_j)} \] with respect to the
  given basis. The nondegeneracy of the form, which we have from
  \Cref{trace-form-bilinear}, is equivalent to the statement that $\det M \ne
  0$, whence the claim follows.
\end{proof}

\begin{lemma}
  Let $(\alpha_i)$ be a basis of $\ringExtension L K$ contained in $B$, with $d
  = d(\alpha_1,\ldots,\alpha_n)$. Then \[ dB \subseteq A\alpha_1 + \cdots +
  A\alpha_n. \]
\end{lemma}

\begin{prop}
  If $\ringExtension L K$ is separable and $A$ is a PID, every finitely
  generated $B$-submodule $M\ne 0$ of $L$ s a free $A$-module of rank $[L:K]$.
\end{prop}
\begin{corollary}
  $B$ admits an integral basis over $A$.
\end{corollary}

\begin{prop}
  Let $\ringExtension M K$ and $\ringExtension N K$ be two Galois extensions
  with $M \cap N = K$, with $m = [M:K]$ and $n = [N:K]$. Fix integral bases
  $(\alpha_i)_{1 \le i \le m}$ of $\ringExtension M K$ and $(\beta_j)_{1 \le j
  \le n}$ of $\ringExtension N K$ respectively, with discriminants $\mu$ and
  $\nu$ respectively. If $\mu$ and $\nu$ are relatively prime, with $x\mu+y\nu
  = 1$ for some $x,y\in A$, then $(\alpha_i\beta_j)$ is an integral basis of
  $MN$, with discriminant $m^\nu n^\mu$.
\end{prop}

% \begin{exercise}
%   foo
% \end{exercise}

\begin{prop}
  If $\fI \subseteq \fJ$ are two nonzero finite $\ko_K$-submodules of $K$, then
  $(\fJ:\fI)$ is finite. Moreover,
  \[ d(\fI) = (\fJ:\fI)^2d(\fJ) \] holds.
\end{prop}
