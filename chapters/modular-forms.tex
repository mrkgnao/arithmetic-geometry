\chapter{Modular forms}

\section{The hyperbolic plane}
\label{sec:the-hyperbolic-plane}

\begin{definition}
  The \emph{upper half-plane} in $\bc$ is
  \[ \upperhalfplane := \{ h\in\bc : Im(h) > 0 \} \]
\end{definition}

\section{M\"obius transformations}
\label{sec:fractional-linear-transformations}

\[ \frac{az+b}{cz+d} \]

\section{The modular group}
\label{sec:the-modular-group}

Define M\"obius transformations

\begin{align*}
  S &= \begin{bmatrix}               & 1 \\ -1 &   \end{bmatrix} \\
  T &= \begin{bmatrix} \hphantom{-}1 & 1 \\    & 1 \end{bmatrix}
\end{align*}

As before, the actions of these two matrices are as follows:

\begin{alignat*}{3}
  & Sz &&= \frac{0z + 1}{-1z + 0} &&= -\frac1z\\
  & Tz &&= \frac{1z + 1}{0z + 1} &&= z + 1
\end{alignat*}

$S$ is an inversion about the unit circle ($z \mapsto 1/z$) followed by
reflection across the imaginary axis ($z \mapsto -z$), while $T$ is a simple
translation.

These form a ``basis'', a generating set, for the modular group:

\begin{prop}
  $PSL_2(\bz) = \langle {S, T} \rangle$.
\end{prop}

% TODO add generators proof from:
% http://math.mit.edu/~brubaker/Math784/modular.pdf

\section{A fundamental domain for $PSL_2(\bz)$}
\label{sec:a-fundamental-domain-for-psl2-Z}

\begin{definition}
  Let $F \subset \upperhalfplane$ be a closed set with connected interior, and
  let $\Gamma$ be a subgroup of $PSL_2(\bz)$. We say $F$ is a \emph{fundamental
    domain} for $\Gamma \backslash \upperhalfplane$ or for $\Gamma$ if
  \begin{enumerate}
  \item any $h \in \upperhalfplane$ is $\Gamma$-equivalent to some point in $F$
  \item no two interior points of $F$ are equivalent under the $\Gamma$ action
  \item the boundary of $F$ is piecewise smooth
  \end{enumerate}
\end{definition}

Define $\bm = PSL_2(\bz)$.

We now exhibit a fundamental domain for $PSL_2(\bz)$. Let
\[F =  \{h \in \upperhalfplane : |\Re(h)| \le \frac12, |h| \ge 1 \}\]

\begin{prop}
  $F$ is a fundamental domain for $\bm$.
\end{prop}

\section{Congruence subgroups}
\label{sec:congruence-subgroups}

\begin{definition}
  Let $N \in \bz_{>0}$. The \emph{modular group} of \emph{level} $N$ is
  \[\Gamma_0(N) = \left\{\begin{bmatrix} a&b\\c&d \end{bmatrix} \in \bm : c
      \equiv 0 \Mod N \right\}\]
\end{definition}

We also have

  \[\Gamma_1(N) = \left\{\begin{bmatrix} a&b\\c&d \end{bmatrix} \in \bm
      : \begin{bmatrix} a&b\\c&d \end{bmatrix}
      \equiv \begin{bmatrix}1&b\\0&1 \end{bmatrix} \Mod N \right\}\]

  and the \emph{principal congruence subgroups}

  \[\Gamma(N) = \left\{\begin{bmatrix} a&b\\c&d \end{bmatrix} \in \bm
      : \begin{bmatrix} a&b\\c&d \end{bmatrix}
      \equiv \begin{bmatrix}1&0\\0&1 \end{bmatrix} \Mod N \right\}\]
