\chapter{Affine schemes}

\section{Motivation}%
\label{sec:motivation-for-affine-schemes}

The main objects of study in ``modern'' algebraic geometry are \emph{schemes} --- a
certain kind of ``geometric space'' that encompasses both geometric questions
(such as questions about lines on surfaces) and number-theoretic or arithmetic
problems (such as questions about integral points on curves). This all works out
because schemes are built from rings, so we can cast geometric problems in terms
of rings of functions on geometric spaces --- and algebraic number theory has
been a thing for a very long time. This is a major idea behind algebraic
geometry: sometimes it pays to look at some geometric object by considering the
functions defined on that space.

\medskip\noindent For instance, given a topological space $T$, you can ask

\begin{question}
  What continuous real-valued functions $f:T\to\br$ can I define on the space,
  satisfying $P$?
\end{question}
where $P$ is some property of continuous $\br$-valued functions.

\medskip\noindent This question, by allowing you an indirect look at the space,
often reveals very interesting information about the space that would have been
difficult to gain otherwise. You can do similar things for smooth or complex
manifolds. This "indirect viewpoint" is the primary one adopted in algebraic
geometry.

\begin{slogan}
  Instead of looking at a space, look at the (rings of) functions defined on its
  subsets.
\end{slogan}

Here is some motivation for what schemes are like, using some words we haven't
defined here yet. You may have heard of differentiable or smooth manifolds: they
are spaces that locally "look like" $\br^n$. For instance, a sphere looks like
$\br^2$ if you "zoom in" enough. And we understand $\br^2$ just fine, and
$\br^n$ in general -- this is just multivariable calculus. The idea here is that
we can work with complicated spaces as long as, "locally", they look like things
we understand.

\medskip\noindent The definition we are aiming for is this:

\begin{adefinition}[Target]
  A scheme is a "locally ringed space" that is locally isomorphic to an affine
  scheme.
\end{adefinition}

We will now spend some time constructing the notions that go into that
statement, building intuition for what these things \emph{are}. In this chapter,
we define \emph{affine schemes}, which are to schemes what Euclidean spaces are
to manifolds.

\section{The spectrum of a ring}%
\label{sec:the-spectrum-of-a-ring}

\begin{adefinition}[Provisional]
 Define the \emph{spectrum} of a ring $R$ as (at least for now) the set
\[ \Spec R = \{ I \subseteq R : I \text{ is a prime ideal} \} \]
\end{adefinition}

The element of $\Spec R$ corresponding to the prime ideal $P$ of $R$ will be
denoted $[P]$ when it is not clear from context.

\section{Some examples}
\label{sec:examples-of-affine-schemes}

\begin{example}
  The ring of \newTerm{dual numbers} over the field $k$ is written
  $\ringOfDualNumbersOver k$. Consider the associated affine scheme

  \[ X = \Spec \dualk \]

  \medskip\noindent What are its points? This is equivalent to asking what the
  prime ideals of $k[\epsilon]/(\epsilon^2)$ are. Since $\epsilon^2 = 0$, $\dualk$ is not an integral domain
  and so $(0)$ is not a prime ideal. Any other ideals will be generated by
  linear polynomials in $\epsilon$, and one can try to reason further by
  considering generators of ideals in $\dualk$, which will be (monic) linear
  polynomials.

  \medskip\noindent However, a quicker argument goes as follows: note that, by
  \Cref{prop:inclusion-preserving-bijection-quotient}, prime ideals of $\dualk$
  are the same as prime ideals of $k[\epsilon]$ containing the ideal
  $(\epsilon^2)$. Since $k[\epsilon]$ is a PID, every prime ideal $\hp$ is
  generated by some prime element $p$. So

  \[ (\epsilon^2) \subset (p) = \hp \implies \divides p {\epsilon^2} \]

  \medskip\noindent
  but the only prime element dividing $\epsilon^2$ is
  $\epsilon$, so $\hp = (\epsilon)$. Hence the only point of the scheme $X$ is
  that corresponding to $(\epsilon)$.

\end{example}
