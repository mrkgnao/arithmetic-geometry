\documentclass[../main.tex]{subfiles}
\chapter{Complex analysis}

\section{Holomorphy and complex differentiability}
\label{sec:holomorphy}

\newcommand{\otobc}{\Omega\to\bc}

There are, broadly speaking, two criteria that we would like nice complex-valued
functions to satisfy. The first is a notion of differentiability similar to the
one from calculus, where every function can be linearly approximated by a
\emph{derivative}, while the second asks that every function be locally
representable by a \emph{power series} expansion.

\medskip\noindent This section will develop these notions, demonstrate relations
between the two, and discuss some simple consequences of these conditions.

\subsection{Initial definitions}

\medskip\noindent Let $\Omega \subseteq \bc$ be an open set.

\begin{definition}
  \label{def:c-diff}
  A function $f : \Omega \to \bc$ is \emph{complex differentiable at $z_0$} if
  the limit
  \[
    f'(z_0) = \lim_{h\to0} \frac{f(z_0+h)-f(z_0)} h
  \]
  exists. $f$ is said to be \emph{complex differentiable on $\Omega$} if it is
  complex differentiable at all $z_0 \in \Omega$.
\end{definition}

\begin{definition}[Narasimhan]
  \label{def:holomorphic}
  $f : \Omega\to\bc$ is \emph{holomorphic on $\Omega$} if, for all $z_0
  \in\Omega$, there exists a neighborhood $U\subseteq\Omega$ of $z_0$ and a
  sequence $\{c_n\}_{n \ge 0}$ of complex numbers such that, for all $z \in U$,
  the series
  \[\sum_{n=0}^\infty c_n (z-z_0)^n\]
  converges to $f(z)$.
\end{definition}

These two definitions are in fact equivalent: holomorphy on $\Omega$ is the same
as $\bc$-differentiability on $\Omega$. This is the content of the
Cauchy-Goursat theorem, which we will prove later (TODO ref).

\subsection{Properties}
\medskip
Holomorphy and complex differentiability imply relations between the
``$x$-behavior'' and ``$y$-behavior'' of a function, so that there are certain
rigidity properties we can be assured of. We now show a few properties which are
all roughly similar in nature, culminating in the \Cref{def:cauchy-riemann}.

\begin{prop}
  \label{thm:partial-x-y-c-diff}
  Let $f : \otobc$ be $\bc$-differentiable at $a\in\Omega$. Then $\partial_x
  f(a)$ and $\partial_yf(a)$ exist, and
  \[\pfrac fx(a) = -i\pfrac fy(a) = f'(a)\]
  holds.
\end{prop}
\begin{proof}
  In the Riemann tradition, write $a = \sigma+it$. We will calculate $f'(a)$ in
  two ways, by approaching $0$ along the real axis, then along the imaginary
  axis. 

  \medskip\noindent
  Taking $0\ne\xi\in\br$,
  \begin{align*}
    f'(a) &= \lim_{\xi\to0}\frac{f(a+\xi)-f(a)}\xi\\
          &=\lim_{\xi\to0}\frac{f(\sigma+\xi,t)-f(\sigma,t)}\xi\\
          &=\pfrac fx(a).
  \end{align*}

  Taking $0\ne\eta\in\br$,
  \begin{align*}
    f'(a) &= \lim_{\xi\to0}\frac{f(a+\eta)-f(a)}{i\eta}\\
          &=\lim_{\xi\to0}\frac{f(\sigma,t+\eta)-f(\sigma,t)}{i\eta}\\
          &=\frac1i \pfrac fy(a).
  \end{align*}

  Equating these two expressions to $f'(a)$ is then enough.
\end{proof}

Note that $x$ and $y$ can be expressed in terms of $z$ and $\bar z$:

\noindent
\begin{align*}
  x &= \frac{z+\bar z}2\\
  y &= \frac{z-\bar z}{2i}
\end{align*}

This means one can (formally?) write, using the chain rule,

\[ \pfrac fz = \pfrac fx \pfrac xz + \pfrac fy \pfrac yz = \frac12\cdot\pfrac fz +
  \frac1{2i}\cdot\pfrac fy = \frac12 (f_z - if_y) \]

\begin{exercise}
  What is the analogous expression for $\partial_{\bar z}$?
\end{exercise}

This motivates the following definition.

\begin{definition}
  \label{def:wirtinger}
  The \emph{Wirtinger derivatives} are differential operators defined as
  follows:

  \noindent
  \begin{alignat*}{3}
    &\partial_z       &&= \pfrac\relax z &&= \frac12\left( \partial_x - i\partial_y \right)\\
    &\partial_{\bar z} &&= \pfrac\relax {\bar z} &&= \frac12\left( \partial_x + i\partial_y \right)
  \end{alignat*}
\end{definition}

\begin{prop}
  If $f:\otobc$ is $\bc$-differentiable at $a\in\Omega$,
  \begin{align*}
    \pfrac fz(a) &= f'(a)\\
    \pfrac f{\bar z}(a) &= 0
  \end{align*}
\end{prop}

\begin{exercise}
  Prove this. (This is essentially a restatement of \Cref{thm:partial-x-y-c-diff}
  using the new notation.)
\end{exercise}


\noindent
\begin{definition}
  \label{def:cauchy-riemann}
  Let $f:\otobc$ be written as $f = u + iv$, where $u, v:\Omega\to\br$. Then the
  equations
  \begin{align*}
    \pfrac fx &= i\pfrac fy\\
    \pfrac fz &= \pfrac fx\\
    \pfrac f{\bar z} &= 0
  \end{align*}

  are each equivalent to the following pair of equations:
  \begin{alignat}{2}
    &u_x &&= v_y\\
    -&u_y &&= v_x
  \end{alignat}
  These differential equations are called the \emph{Cauchy-Riemann equations}.
\end{definition}

% Note that

% \noindent
% \begin{alignat}{3}
%   &x &&= \frac12 &&({z+\bar z})\\
%   &y &&= \frac1{2i} &&({z-\bar z})
% \end{alignat}

% Then the action of the Wirtinger derivatives is seen to really be equivalent to


% From the explicit limits we calculated above, we know that holomorphy on a
% domain $U$ implies that the Cauchy-Riemann equations hold on $U$. In fact, the
% converse is also true!

% Holomorphy is a much more rigid requirement than real-analytic differentibility
% on $\br^2$.

Define an $\br$-isomorphism of fields

\medskip\noindent
\begin{alignat*}{3}
  &\mu&&:\bc&&\to\br^2\\
      &&&x + iy&&\mapsto(x,y)
\end{alignat*}

Let $f:\otobc$ have first partial derivatives at $w$. We have the
\emph{Jacobian map}, represented in the standard basis by

\[
J_w(u, v) = \begin{bmatrix}
 u_x(w) & u_y(w)\\v_x(w) & v_y(w) 
\end{bmatrix}
\]

\medskip\noindent
This is a local isomorphism of $\br^2$ onto the tangent space $T_w\br^2 \simeq
\br^2$. We ``lift'' this to $\bc$:

\begin{definition}
  The \emph{tangent map} of $f = u + iv$ at $w$ is
  \[d_wf := \mu^{-1} \triangleleft J_w(u,v) \triangleleft \mu \]
\end{definition}

\begin{prop}
  We have $\partial_{\bar z}f(w) = 0$ iff $d_wf$ is $\bc$-linear, that is, if
  \[d_wf(\lambda\cdot z) = \lambda\cdot d_wf(z)\]
  in which case
  \[d_wf(z) = z \cdot \partial_z f(w) = z \cdot f'(w) \]
\end{prop}

Notice that this says exactly that $f$ is locally linear.

\begin{proof}
  TODO. Pretty weird in Narasimhan.
\end{proof}

$\bc$-differentiable functions satisfy the expected properties:
\begin{enumerate}
\item Given differentiable $f,g:\otobc$ and $\lambda\in\bc$,
  \begin{align*}
    f + g &: z \mapsto f(z) + g(z)\\
    f\cdot g &: z\mapsto f(z) \cdot g(z)\\
    \lambda\cdot f &:z\mapsto \lambda \cdot f(z)
  \end{align*}
  are all $\bc$-differentiable.
\end{enumerate}

\section{Zeroes and poles}
\label{sec:zeroes-and-poles}

\begin{definition}[Zeros]

\end{definition}

\begin{definition}
  A function $f : \bc \to \bc$ is \emph{elliptic} if, for all $\lambda$ in some
  lattice $\Lambda$, $f(z + \lambda) = f(z)$ for all $z \in \bc$.
\end{definition}

\begin{theorem}[Liouville]
  Any bounded entire function is constant.
\end{theorem}

\begin{definition}
  \label{def:entire}
  If $f$ is holomorphic on all of $\bc$, it is said to be \emph{entire}.
\end{definition}
