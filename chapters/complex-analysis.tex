\documentclass[../main.tex]{subfiles}
\chapter{Complex analysis}

\section{Holomorphy and complex differentiability}
\label{sec:holomorphy}

\newcommand{\otobc}{\Omega\to\bc}

There are, broadly speaking, two criteria that we would like nice complex-valued
functions to satisfy. The first is a notion of differentiability similar to the
one from calculus, where every function can be linearly approximated by a
\emph{derivative}, while the second asks that every function be locally
representable by a \emph{power series} expansion.

\medskip\noindent This section will develop these notions, demonstrate relations
between the two, and discuss some simple consequences of these conditions.

\subsection{Initial definitions}

\medskip\noindent Let $\Omega \subseteq \bc$ be an open set.

\begin{definition}
  \label{def:c-diff}
  A function $f : \Omega \to \bc$ is \emph{complex differentiable at $z_0$} if
  the limit
  \[
    f'(z_0) = \lim_{h\to0} \frac{f(z_0+h)-f(z_0)} h
  \]
  exists. $f$ is said to be \emph{complex differentiable on $\Omega$} if it is
  complex differentiable at all $z_0 \in \Omega$.
\end{definition}

\begin{definition}[Narasimhan]
  \label{def:holomorphic}
  $f : \Omega\to\bc$ is \emph{holomorphic on $\Omega$} if, for all $z_0
  \in\Omega$, there exists a neighborhood $U\subseteq\Omega$ of $z_0$ and a
  sequence $\{c_n\}_{n \ge 0}$ of complex numbers such that, for all $z \in U$,
  the series
  \[\sum_{n=0}^\infty c_n (z-z_0)^n\]
  converges to $f(z)$.
\end{definition}

These two definitions are in fact equivalent: holomorphy on $\Omega$ is the same
as $\bc$-differentiability on $\Omega$. This is the content of the
Cauchy-Goursat theorem, which we will prove later (TODO ref).

\subsection{Properties}
\medskip Holomorphy and complex differentiability imply relations between the
``$x$-behavior'' and ``$y$-behavior'' of a function, so that there are certain
rigidity properties we can be assured of. We now show a few properties which are
all roughly similar in nature, culminating in \Cref{def:cauchy-riemann}.

\begin{prop}
  \label{thm:partial-x-y-c-diff}
  Let $f : \otobc$ be $\bc$-differentiable at $a\in\Omega$. Then $\partial_x
  f(a)$ and $\partial_yf(a)$ exist, and
  \[\pfrac fx(a) = -i\pfrac fy(a) = f'(a)\]
  holds.
\end{prop}
\begin{proof}
  In the Riemann tradition, write $a = \sigma+it$. We will calculate $f'(a)$ in
  two ways, by approaching $0$ along the real axis, then along the imaginary
  axis.

  \medskip\noindent Taking $0\ne\xi\in\br$,
  \begin{align*}
    f'(a) &= \lim_{\xi\to0}\frac{f(a+\xi)-f(a)}\xi\\
          &=\lim_{\xi\to0}\frac{f(\sigma+\xi,t)-f(\sigma,t)}\xi\\
          &=\pfrac fx(a).
  \end{align*}

  Taking $0\ne\eta\in\br$,
  \begin{align*}
    f'(a) &= \lim_{\xi\to0}\frac{f(a+\eta)-f(a)}{i\eta}\\
          &=\lim_{\xi\to0}\frac{f(\sigma,t+\eta)-f(\sigma,t)}{i\eta}\\
          &=\frac1i \pfrac fy(a).
  \end{align*}

  Equating these two expressions to $f'(a)$ is then enough.
\end{proof}

Note that $x$ and $y$ can be expressed in terms of $z$ and $\bar z$:

\noindent
\begin{align*}
  x &= \frac{z+\bar z}2\\
  y &= \frac{z-\bar z}{2i}
\end{align*}

This means one can (formally?) write, using the chain rule,

\[ \pfrac fz = \pfrac fx \pfrac xz + \pfrac fy \pfrac yz = \frac12\cdot\pfrac fz
  + \frac1{2i}\cdot\pfrac fy = \frac12 (f_z - if_y) \]

\begin{exercise}
  What is the analogous expression for $\partial_{\bar z}$?
\end{exercise}

This motivates the following definition.

\begin{definition}
  \label{def:wirtinger}
  The \emph{Wirtinger derivatives} are differential operators defined as
  follows:

  \noindent
  \begin{alignat*}{3}
    &\partial_z       &&= \pfrac\relax z &&= \frac12\left( \partial_x - i\partial_y \right)\\
    &\partial_{\bar z} &&= \pfrac\relax {\bar z} &&= \frac12\left( \partial_x +
      i\partial_y \right)
  \end{alignat*}
\end{definition}

\begin{prop}
  If $f:\otobc$ is $\bc$-differentiable at $a\in\Omega$,
  \begin{align*}
    \pfrac fz(a) &= f'(a)\\
    \pfrac f{\bar z}(a) &= 0
  \end{align*}
\end{prop}

\begin{exercise}
  Prove this. (This is essentially a restatement of
  \Cref{thm:partial-x-y-c-diff} using the new notation.)
\end{exercise}


\noindent
\begin{definition}
  \label{def:cauchy-riemann}
  Let $f:\otobc$ be written as $f = u + iv$, where $u, v:\Omega\to\br$. Then the
  equations
  \begin{align*}
    \pfrac fx &= i\pfrac fy\\
    \pfrac fz &= \pfrac fx\\
    \pfrac f{\bar z} &= 0
  \end{align*}

  are each equivalent to the following pair of equations:
  \begin{alignat}{2}
    &u_x &&= v_y\\
    -&u_y &&= v_x
  \end{alignat}
  These differential equations are called the \emph{Cauchy-Riemann equations}.
\end{definition}

% Note that

% \noindent
% \begin{alignat}{3}
%   &x &&= \frac12 &&({z+\bar z})\\
%   &y &&= \frac1{2i} &&({z-\bar z})
% \end{alignat}

% Then the action of the Wirtinger derivatives is seen to really be equivalent
% to


% From the explicit limits we calculated above, we know that holomorphy on a
% domain $U$ implies that the Cauchy-Riemann equations hold on $U$. In fact, the
% converse is also true!

% Holomorphy is a much more rigid requirement than real-analytic
% differentibility on $\br^2$.

Define an $\br$-isomorphism of fields

\medskip\noindent
\begin{alignat*}{3}
  &\mu&&:\bc&&\to\br^2\\
  &&&x + iy&&\mapsto(x,y)
\end{alignat*}

Let $f:\otobc$ have first partial derivatives at $w$. We have the \emph{Jacobian
  map}, represented in the standard basis by

\[
  J_w(u, v) = \begin{bmatrix} u_x(w) & u_y(w)\\v_x(w) & v_y(w)
  \end{bmatrix}
\]

\medskip\noindent This is a local isomorphism of $\br^2$ onto the tangent space
$T_w\br^2 \simeq \br^2$. We ``lift'' this to $\bc$:

\begin{definition}
  The \emph{tangent map} of $f = u + iv$ at $w$ is
  \[d_wf := \mu^{-1} \triangleleft J_w(u,v) \triangleleft \mu \]
\end{definition}

\begin{prop}
  We have $\partial_{\bar z}f(w) = 0$ iff $d_wf$ is $\bc$-linear, that is, if
  \[d_wf(\lambda\cdot z) = \lambda\cdot d_wf(z)\] in which case
  \[d_wf(z) = z \cdot \partial_z f(w) = z \cdot f'(w) \]
\end{prop}

Notice that this says exactly that $f$ is locally linear.

\begin{proof}
  TODO. Pretty weird in Narasimhan.
\end{proof}

$\bc$-differentiable functions satisfy the expected properties:
\begin{enumerate}
\item Given differentiable $f,g:\otobc$ and $\lambda\in\bc$,
  \begin{align*}
    f + g &: z \mapsto f(z) + g(z)\\
    f\cdot g &: z\mapsto f(z) \cdot g(z)\\
    \lambda\cdot f &:z\mapsto \lambda \cdot f(z)
  \end{align*}
  are all $\bc$-differentiable.
\item Consider open sets $U,V$ in $\bc$. If $f: U\to\bc$ and $g:V\to\bc$ are
  complex differentiable, then $g\triangleleft f : U\to\bc$ is complex
  differentiable if it is defined --- that is, if $f(U)\subseteq V$. Further,
  for $z_0\in U$, we have a \emph{chain rule}:
  \[ (g\triangleleft f)' = (g'\triangleleft f)\cdot f' \]
\end{enumerate}

Recall that a function is $C^k$ on some domain if all partial derivatives of
orders $\le k$ exist and are continuous.

\begin{prop}
  \label{prop:c-r-implies-c-diff}
  Let $f:\otobc$ be $C^1$ and satisfy the Cauchy-Riemann equations on $\Omega$.
  Then $f$ is $\bc$-differentiable on $\Omega$.
\end{prop}

In fact, the partial derivatives only need to exist: the Looman--Menchoff
theorem (TODO ref) says that they need not be assumed to be continuous
themselves.

\begin{proof}
  Let $\zeta = \xi + i\eta$. Write $f = u + iv$, and let $w = \alpha + i\beta
  \in \Omega$. Using Taylor's theorem for $C^1$ functions,
  \begin{align*}
    u(w+\zeta)-u(w) &= \tilde u(\alpha+\zeta, \beta+\eta) - \tilde u(\alpha,\beta)\\
                    &= \pfrac {\tilde u}x(\alpha,\beta)\cdot\xi + \pfrac {\tilde u}y(\alpha,\beta)\cdot\eta + \varepsilon_1(\xi,\eta)\\
                    &= \pfrac ux(w)\cdot\xi + \pfrac uy(w)\cdot\eta + \varepsilon_1(\xi,\eta)
  \end{align*}
  
  Similarly, we have
  \[ v(w+\zeta) - v(w) = \pfrac ux(w)\cdot\xi + \pfrac uy(w)\cdot\eta +
    \varepsilon_2(\xi,\eta)\]

  Crucially, as $\xi, \eta\to 0$, we have the bounds
  \begin{align*}
    \frac{\varepsilon_1(\xi, \eta)}{|\xi| + |\eta|} &\to 0\\
    \frac{\varepsilon_2(\xi, \eta)}{|\xi| + |\eta|} &\to 0\\
    \frac{\varepsilon(\xi, \eta)}{|\zeta|}  &\to 0 (how?)
  \end{align*}

  where $\varepsilon = \varepsilon_1 + i\varepsilon_2$.

  We combine the previous two equations to get
  \[f(w + \zeta) - f(w) = f_x(w)\cdot\xi + f_y(w)\cdot\eta + \epsilon(\zeta) \]

  and, using the Cauchy-Riemann equations to write $f_y$ as $if_x$, we see that
  the limit
  \begin{align*}
    \lim_{\zeta\to0}\frac{f(w+\zeta)-f(w)}\zeta &= \frac{f_x(w) \cdot (\xi+i\eta)}{\zeta} + \lim_{\zeta\to0}\frac{\epsilon(\zeta)}\zeta\\
                                                &=\pfrac fx(w)
  \end{align*}
  exists (and is differentiable?).
\end{proof}

\begin{erratum}
  In his proof of \Cref{prop:c-r-implies-c-diff}, where Narasimhan writes $\zeta
  = \mathcolor{blue}\zeta + i\eta$, he means $\zeta = \mathcolor{red}{\xi} +
  i\eta$.
\end{erratum}

\section{Power series}
\label{sec:power-series}

\begin{lemma}[Abel]
  Given a sequence $\{c_n\}_{n\ge0}$ in $\bc$, there exists $0\le R \in
  \br_\infty = \br \cup \{\infty\}$ such that the series
  \[\sum_{n=0}^\infty c_nz^n\]
  converges for $|z|<R$ and diverges for $|z|>R$. Further, the series converges
  \textbf{uniformly} on any compact subset of $D_R(0)$.
\end{lemma}

This means that, given any power series, there exists a disc inside which it
converges and outside which it diverges. Abel's lemma does not say what will
happen on the boundary circle $|z| = R$ (funny things do sometimes).

\begin{proof}
  Choose
  \[R = \sup\;\{r \ge 0 : \exists M\ldotp\forall n\ge 0\ldotp|c_n|r^n \le M
    \} \] or, in words, $R$ is the ``biggest'' disc inside which $|c_n||z|^n$ is
  bounded. By construction,
  \[|z| > R \implies |c_n||z|^n \text{ is not bounded} \] so (why? TODO) $\sum
  c_nz^n$ cannot converge.

  \medskip\noindent Let $K\subseteq D_R$ be compact. Choose $\rho < R$ such that
  $K \subseteq D_\rho$, and $r$ such that $\rho < r < R$. There exists (?) $M>0$
  such that $|c_n|r^n \le M$. For $z\in K$,

  \medskip\noindent
  \[ |c_nz^n| \le |c_n|\rho^n \le M \left( \frac\rho r \right)^n \]

  Since $\rho < r$, we have $M\sum\left( \frac\rho r \right)^n < \infty$, so
  that $\sum c_nz^n$ converges uniformly on $K$.
\end{proof}

\begin{definition}
  The real number $R$ is called the \emph{radius of convergence} of
  $\sum_{n=0}^\infty c_nz^n$.
\end{definition}

\begin{corollary}
  Holomorphy on an open set $\Omega\subseteq\bc$ implies continuity on $\Omega$.
\end{corollary}

\begin{lemma}
  The radius of convergence of
  \[\sumoi n c_n z^{n-1}\]
  is equal to that of
  \[\sumzi c_n z^n.\]
\end{lemma}
\begin{proof}

\end{proof}

\subsubsection{Derivatives of power series}

\begin{lemma}%
  \label{lemma:power-difference-bound}
  Let $a, b \in\bc$. We have
  \[\abs {(a+b)^n - a^n} \le n \abs b \left( \abs a + \abs b \right)^{n-1}\] for
  $n\ge1$.
\end{lemma}

\begin{proof}
  From high-school algebra, we have the factorization
  \begin{align*}
    (a+b)^n - a^n &= (a+b-a)\sum_{k=0}^{n-1}(a+b)^{n-1-k}-a^k\\
                  &= b\sum_{k=0}^{n-1}(a+b)^{n-1}\left(\frac{a}{a+b}\right)^k
  \end{align*}
  We take absolute values of both sides and apply the triangle inequality:
  \begin{align*}
    \abs{(a+b)^n - a^n} &= {\abs b}\abs{\sum_{k=0}^{n-1}{(a+b)}^{n-1}\left(\frac{a}{a+b}\right)^k} \\
                        &\le {\abs b}\sum_{k=0}^{n-1}{\abs{a+b}}^{n-1}\abs{\frac{a}{a+b}}^k \\
                        &\le {\abs b}\sum_{k=0}^{n-1}{\abs{a+b}}^{n-1} \\
                        &= n {\abs b}{\left(  \abs{a} + \abs{b}\right)}^{n-1}
  \end{align*}
\end{proof}

\begin{prop}
  Let $R>0$, and consider a power series
  \begin{align*}
    \sumzi {c_n(z-z_0)^n} \to f(z) &\;\;\text{ for } z \in D_R(z_0)
  \end{align*}
  Then $f$ is $\bc$-differentiable on $D_R(z_0)$ and
  \[ f'(z) = \sumoi {nc_n{(z-z_0)}^{n-1}} \text{ for } z\in D_R(z_0) \]
\end{prop}

\begin{proof}
  Let $z \in D_R(z_0)$. Set

  \[ R_\pm = \frac12(R\pm\abs{z-z_0})\]

  \medskip\noindent
  and choose a complex number $\zeta$ satisfying $0 < \abs \zeta < R_-$. Then

  \[ \frac{f(z+\zeta) - f(z)}{\zeta} = \sumoi{c_n\frac{{(z+\zeta-z_0)}^n-{(z-z_0)}^n}{\zeta}} \]
  
  By \Cref{lemma:power-difference-bound},
  \begin{align*}
    \abs{\sum_{n>N}{c_n\frac{{(z+\zeta-z_0)}^n-{(z-z_0)}^n}{\zeta}}} &\le \sum_{n>N}n\abs{c_n}{(\abs\zeta+\abs{z-z_0})}^{n-1} \\
                                                          &\le \sum_{n>N}n\abs{c_n}R_+^{n-1}
  \end{align*}
\end{proof}

\section{Zeroes and poles}%
\label{sec:zeroes-and-poles}

\begin{definition}[Zeros]

\end{definition}

\begin{definition}
  A function $f : \bc \to \bc$ is \emph{elliptic} if, for all $\lambda$ in some
  lattice $\Lambda$, $f(z + \lambda) = f(z)$ for all $z \in \bc$.
\end{definition}

\begin{theorem}[Liouville]
  Any bounded entire function is constant.
\end{theorem}

\begin{definition}
  \label{def:entire}
  If $f$ is holomorphic on all of $\bc$, it is said to be \emph{entire}.
\end{definition}
