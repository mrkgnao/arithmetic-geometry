\documentclass[../main.tex]{subfiles}
\chapter{Complex analysis}

\section{Holomorphy}
\label{sec:holomorphy}

In the definition of the derivative

\begin{equation*}
  f'(z) = \lim_{h\to0} \frac{f(x+h)-f(x)} h
\end{equation*}

take $h\in\bc$. Write

\begin{alignat*}{3}
  &f(z) &&= u(z) &&+ iv(z)\\
  &f(x,y) &&= u(x,y) &&+ iv(x,y)
\end{alignat*}

Then $f' = u_x + iv_x$, and similarly taking $h = ik$ gives $f' = v_y - iu_y$.
Comparing the two expressions, we have

\begin{alignat}{2}
  &u_x &&= v_y\\
  -&u_y &&= v_x
\end{alignat}

\begin{definition}
  A function $f : \bc \to \bc$ is \emph{holomorphic}, or \emph{complex
    differentiable}, 
\end{definition}

\section{Zeroes and poles}
\label{sec:zeroes-and-poles}

\begin{definition}[Zeros]

\end{definition}
