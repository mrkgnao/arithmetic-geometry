\chapter{Category theory}

\section{(Co)limits}

\begin{definition}
  An \emph{initial object} in $\rc$ is a limit of the empty diagram in $\rc$.
\end{definition}

\begin{definition}
  A \emph{final object} in $\rc$ is a colimit of the empty diagram in $\rc$.
\end{definition}

\section{Adjunctions}

Consider a pair of functors between $\rc$ and $\rd$.
\begin{align*}
  F \colon &\rc \to        \rd\\
           &\rc \leftarrow \rd \colon G
\end{align*}

\begin{definition}
  An adjunction between $\rc$ and $\rd$ consists of natural transformations
  \begin{align*}
    \eta \colon 1_\rc \to &G\triangleleft F\\
                          &F\triangleleft G \to 1_\rd \colon \epsilon
  \end{align*}
  (called the \emph{unit} $\eta$ and the \emph{counit} $\epsilon$ respectively).
\end{definition}

$F$ and $G$ are said to form an \emph{adjoint pair}, and $F$ is said to be
\emph{left adjoint} to $G$, written
\[ F\dashv G \]
(and $G$ is \emph{right adjoint} to $F$).

\begin{definition}
  Another definition states that $F$ and $G$ form an adjoint pair if there is an
  isomorphism

  \[ \rc(GY, X) \simeq \rd(Y, FX) \]

  natural in $X\in\rc$ and $Y\in\rd$.
\end{definition}
